\chapter{Staphylococcus aureus}

\emph{Staphylococcus aureus} is a gram positive (which means that it has a peptidoglican layer into the cell wall) and it is a facultative anaerobe bacterium. This is very crucial for \emph{S. aureus} epidemiology because it is able to colonize nostrils where there is oxygen, but it is also able to colonize organs that are inside the body. It is  one of the main players in common food poisoning. It is also involved in the menstrual toxic shock syndrome. It plays a key role, also, in other serious disease, like osteomyelitis which in an infection of the bones or sepsis that is a systemic infections. 
It is a common skin colonizer and for this reason 25$\%$ of the people in this room probably have \emph{S. aureus} in their body, but it is also the cause of very bad skin infection. 
\emph{S. aureus} is tricky to treat because of is immune evasion strategies.

\section{Immune evasion strategies}

\emph{S. aureus} has two different main strategies that used in order to stop the immune system of the host from getting rid of it. 

\begin{enumerate}
    \item It prevents the engagement of the host immune system, so it is not be recognized by the host immune system. That is done by different proteins that are present on the  surface of \emph{S. aureus}. There are four different kind of classes of proteins that are all try to hide the \emph{S. aureus} to the host immune system. 
    
    \begin{itemize}
        \item Adhesins bind complement factors to inhibit complement activation cascade.
        \item Leukocidins are instead a number of toxins that can selectively kill the adapting immune cells, so killing those immune cells that would be able to kill \emph{S. aureus}. 
        \item Immunoglobulin binding proteins bind and immobilize IgGs, so they cannot start the cascade of activation of the immune systems.
        \item Proteases that cleave the immunoglobulin that are responsible for the activation.
    \end{itemize}
    
    \item Overactiovation of the non-specific immune system. It is able to trigger a lot of inflammation to cytokine release, that is a non-specific reaction of the body, and also to facilitate invasion of the so-called non-professional phagocytes, the neutrophiles. There is the production of autholysins, that facilitate invasion of non-professional phagocytes, and super-antigens, that activate T cells and trigger the cytokine release. This could be an advantage to \emph{S. aureus} because when a neutrophiles phagocytes a bacterium or any pathogen can happen that the microbiome is uptaken by the neutrophile, is killed through degranulation and ROS production. The neutrophile then undergoes apoptosis and it is removed by macrophages and so we have a resolution of the infection. 
    If \emph{S. aureus} is present, we have the neutrophile that uptake him. However, \emph{S. aureus} stops the apoptosis of the neutrophile and is able to divide inside it and be screened by the immune system of the host and then, with the leukocidins, it can cause some holes in the neutrophile and also the release of its content outside causing an extra inflammation. 
\end{enumerate}

\section{Antibiotic resistance in \emph{S. aureus}}

In the 1940s we have the introduction of penicillin G that is quickly followed by emergence of resistance, the penicillinase. 
In 1959 was found the methicillin, a semisynthetic penicillinase-resistant $\beta$-lactam antibiotics, but in 1960 we have already some cases of resistance to this antibiotic from an hospital in UK (MRSA). In 1960 MRSA emerged in many countries. 

\subsection{Methicillin-resistant \emph{S. aureus} (MRSA)}

\emph{S. aureus} is resistant to $\beta$-lactam antibiotics and is also able to acquire other resistances, even to last resource antibiotic, like vancomycin, linezolid, daptomycin.

Because \emph{S. aureus} is a gram-positive bacterium, the peptidoglycan layer of the cell wall is extremely important for the correct assembly of the cell membrane and the $\beta$-lactam antibiotics have the ability to acts as substrate analog causing an impaired transpeptidation of the peptidoglycan and the creation of a defective cell wall during cell division.

\begin{enumerate}
    \item In absence of $\beta$-lactam antibiotics, we have the normal cell-wall biosynthesis.
    \item In presence of $\beta$-lactam antibiotics, we have the binding of the antibiotic to the PBP active site and therefore the peptidoglycan cannot be transpeptidated and the peptidoglycan layer of the cell wall cannot be produced and so we have the cell death during division. 
    \item In presence of the $\beta$-lactam antibiotics, but with mutated PBP (PBP2a, aka MecA), $\beta$-lactam is not able to bind the modified PBP. So, the peptidoglycan can be normally transpeptidated and \emph{S. aureus} can produce the cell wall and proliferate. 
\end{enumerate}

\subsection{Methicillin resistance: where is it encoded?}

The resistance to methicillin, but more in general to $\beta$-lactam antibiotics, is not encoded on a plasmid. It is encoded on a mobile genetic element that is called staphylococcal chromosome cassette mec (SCCmec). SCCmec are mobile genetic elements that are wide spread across staphylococci genome. They commonly carries genes that might confers some increased fitness for specific environments. This type of mobile genetic elements can be easily integrated into the genome and also can easily excises from the genome. That means that is really easy for a MSSA to integrate the mobile genetic element in case of a strong selective pressure that might be given by the presence of antibiotic. When the antibiotic is not more present, it is easy for MSSA to excise the mobile genetic element and return to the basic state of methicillin asset of the \emph{S. aureus}. 

The mobile genetic element is not maintained inside the cell. It is integrated into the cell or it is excised and released outside.

\subsection{Methicillin-resistant \emph{S. aureus} (MRSA)}
S. aureus is not well recognized for the problems that it causes. There are 80 thousands new patients per year in US that have invasive infections (not colonization), so that means that are people that are sick. The mortality rate is 20$\%$. 
Hospitalize patients, immune compromise patients or patients with conditions like cystic fibrosis are very expose to \emph{S. aureus}. This is way, in 2017, the World Health Organization insert \emph{S. aureus} resistant to methicillin and partially resistant to vancomycin or completely as a high priority bug for the research and the development of new antibiotics. Is the fifth one in the list.
An article of three years ago estimated about 5 millions deaths associated with bacterial antibiotic resistances (not \emph{S. aureus} only) in 2019. Moreover, the World Health Organization has estimated that by 2025 there will be 10 millions deaths per years because of antibiotic resistance. 

\subsubsection{\emph{S.aureus} worldwide}
There are lot of studies that focused on MRSA or \emph{S. aureus} infections, but the problem is that there are quite biases toward specifically lineages. With the term lineages, we refer to a specific strain or a group of strains that are known to be particularly hyper-virulent or resistant or affecting a specific population (e.g. cystic fibrosis patients). So, only a part of the pool of infections that \emph{S. aureus} can cause. There is a greater underestimation of the strains that are sensitive to methicillin. There is no a clear reason for that because if \emph{S. aureus} can be sensitive to methicillin, so it can be resistant to all the other antibiotics. 
There is a great variability in MRSA that can change epidemiology:
\begin{itemize}
    \item 60s-70s. There were lots of hospitals associated infections, so people go to the hospital, get a surgery and get \emph{S. aureus}. Also, nowadays \emph{S. aureus} is a key player in post-surgery infections. 
    \item 80s-90s. The community start talking about community-associated \emph{S. aureus} infections and methicillin resistance of \emph{S. aureus} because the study starting focusing on the dissemination, also in healthy people.
    \item 2000s. There was a great studies on livestock-associated MRSA. Zoonotic infections are pretty relevant, but treatment with antibiotics cause resistances in that community. Livestock diseases and resistances are serious consequences.  
\end{itemize}

\section{Whole genome epidemiology, characterization, and phylogenetic reconstruction of \emph{S. aureus} strains in a pediatric hospitals}

This work is a full pipeline of what you do if you want to do a survey of the general populations of \emph{S. aureus} in a specific place. 
